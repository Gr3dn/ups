\documentclass[11pt,a4paper]{article}

% Encoding and Czech language
\usepackage[utf8]{inputenc}
\usepackage[T1]{fontenc}
\usepackage[czech]{babel}

% Page layout
\usepackage{geometry}
\geometry{margin=2.2cm}
\usepackage{setspace}
\setstretch{1.05}
\setlength{\parindent}{0pt}
\setlength{\parskip}{0.25em}

% Lists
\usepackage{enumitem}

% Links
\usepackage[hidelinks]{hyperref}

% Code listings
\usepackage{xcolor}
\usepackage{listings}
\lstdefinelanguage{C45proto}{
  morekeywords={C45OK,C45WRONG,C45DOWN,C45REC,C45REC_OK,C45PI,C45PO,C45J,C45B,C45H,C45S,C45L,C45D,C45T,C45C,C45TO,C45R,C45OD,C45OB},
  sensitive=true,
}
\lstset{
  basicstyle=\ttfamily\small,
  columns=fullflexible,
  breaklines=true,
  frame=single,
  framerule=0.4pt,
  rulecolor=\color{black!35},
  keywordstyle=\color{blue!60!black}\bfseries,
  commentstyle=\color{black!60},
  stringstyle=\color{teal!60!black},
  showstringspaces=false
}

\begin{document}

\begin{center}
{\LARGE\textbf{Dokumentace semestrální práce}}\\[4pt]
\textbf{Předmět:} KIV/UPS \quad $\bullet$ \quad \textbf{Autor:} Danyil Hrechishkin \quad $\bullet$ \quad \textbf{Datum vytvoření:} 29.01.2026
\end{center}

\begin{center}
\Large\textbf{Blackjack --- TCP klient/server (JavaFX + C)}
\end{center}

\vspace{0.5em}
\textbf Projekt implementuje síťovou dvouhráčovou variantu hry Blackjack. Server (C, POSIX) poskytuje lobbyny a běh hry, klient (Java 17 + JavaFX) poskytuje UI, komunikaci a automatický reconnect při výpadku sítě.

\section{Základní popis hry}

Blackjack je karetní hra, ve které se hráč snaží dosáhnout hodnoty ruky co nejblíže k 21, bez jejího překročení.
Tato implementace je \textbf{dvouhráčová} (bez samostatného dealera). Oba hráči dostanou dvě karty a následně se střídají v tazích.

\subsection{Použitá varianta pravidel}
\begin{itemize}[topsep=0.25em]
  \item Používá se standardní balíček 52 karet.
  \item Hodnoty: 2--10 dle čísla, J/Q/K = 10, eso (A) = 11 nebo 1 (dle toho, aby součet nepřesáhl 21).
  \item Každý hráč začíná se dvěma kartami.
  \item Ve svém tahu hráč volí:
    \begin{itemize}[topsep=0.25em]
      \item \textbf{HIT} --- dobere jednu kartu.
      \item \textbf{STAND} --- ukončí dobírání.
    \end{itemize}
  \item Pokud hráč po HIT překročí 21, nastává \textbf{bust} a pro hráče je jeho skóre nastaveno na -1.
  \item Hra končí, když oba hráči stojí nebo bustnou.
  \item Výsledkem je vítěz (vyšší skóre; -1 je nejhorší) nebo \textbf{PUSH} (remíza).
\end{itemize}

\subsection{Časování a odolnost}
Hra je navržena tak, aby byla použitelná i na nestabilní síti:
\begin{itemize}[topsep=0.25em]
  \item Každý tah má časový limit (na serveru \texttt{TURN\_TIMEOUT\_SEC}).
  \item Klient i server používají keep-alive zprávy (PING/PONG), aby se výpadek odhalil v rozumném čase.
  \item Při odpojení jednoho hráče během hry server čeká omezenou dobu na jeho návrat (\texttt{RECONNECT\_TIMEOUT\_SEC}) druhý hráč je o tom informován.
  \item Klient při ztrátě spojení provádí automatický reconnect v definovaném okně.
\end{itemize}

\section{Přehled řešení a rozvrstvení aplikace}

Projekt je rozdělen na dvě hlavní části:
\begin{itemize}[topsep=0.25em]
  \item \textbf{Server}: správa připojení, handshake, lobby, běh herních vláken, reconnect.
  \item \textbf{Klient}: UI scény, síťová vrstva, parsování protokolu, automatické obnovení spojení.
\end{itemize}

\subsection{Architektonický přehled}
Klient je rozdělen na:
\begin{itemize}[topsep=0.25em]
  \item \textbf{UI vrstvu (JavaFX)} --- scény, ovládání hry, zobrazování stavu spojení,
  \item \textbf{síťovou vrstvu} --- TCP komunikace, parsování protokolu, keep-alive a reconnect.
\end{itemize}

Server je rozdělen na:
\begin{itemize}[topsep=0.25em]
  \item \textbf{síťovou vrstvu} --- accept-loop a per-client vlákna,
  \item \textbf{logiku hry} --- lobby, herní vlákno a běh zápasu,
  \item \textbf{I/O pomocné funkce} --- bezpečné čtení/psaní řádků a serializace snapshotů.
\end{itemize}

\subsection{Adresářová struktura}
\begin{itemize}[topsep=0.25em]
  \item \texttt{server/} --- serverová část v jazyce C
    \begin{itemize}[topsep=0.25em]
      \item \texttt{src/main.c} --- CLI entry point, načtení konfigurace, start serveru
      \item \texttt{src/server.c} --- síťový stavový automat klienta, registry jmen, accept-loop
      \item \texttt{src/game.c} --- lobby + herní logika, herní vlákna, reconnect ve hře
      \item \texttt{src/protocol.c} --- nízkoúrovňové čtení/zápis řádků a serializace snapshotů lobby
      \item \texttt{include/*.h} --- hlavičky (API a datové struktury)
      \item \texttt{config.txt} --- runtime konfigurace (IP, port, počet lobby)
    \end{itemize}
  \item \texttt{clientBlackJack/client/} --- klient v Javě
    \begin{itemize}[topsep=0.25em]
      \item \texttt{com.blackjack.MainApp} --- JavaFX aplikace, scény, navigace a binding na síť
      \item \texttt{com.blackjack.net.NetClient} --- TCP klient, reader thread, parser protokolu, keep-alive a reconnect
      \item \texttt{com.blackjack.net.ProtocolListener} --- callback rozhraní pro UI
      \item \texttt{com.blackjack.ui.GameView} --- herní okno (log, tlačítka, reconnect label)
      \item \texttt{pom.xml} --- Maven konfigurace (Java 17, JavaFX)
    \end{itemize}
\end{itemize}

\subsection{Použité knihovny a prostředí}
\begin{itemize}[topsep=0.25em]
  \item \textbf{Server:}
    \begin{itemize}[topsep=0.25em]
      \item Jazyk: C11
      \item Překladač: \texttt{gcc} (nebo kompatibilní)
      \item POSIX API: \texttt{socket}, \texttt{poll}, \texttt{pthread}, \texttt{recv/send}, \texttt{shutdown}, \texttt{getifaddrs}
    \end{itemize}
  \item \textbf{Klient:}
    \begin{itemize}[topsep=0.25em]
      \item Java: \textbf{17} (Maven \texttt{maven.compiler.release=17})
      \item JavaFX: \textbf{17.0.15} (artefakty \texttt{javafx-controls}, \texttt{javafx-fxml})
      \item Build: Maven + \texttt{javafx-maven-plugin}
    \end{itemize}
\end{itemize}

\section{Protokol (specifikace pro alternativní implementaci)}
\label{sec:protocol}

\subsection{Základní pravidla formátu zpráv}
\begin{itemize}[topsep=0.25em]
  \item Komunikace probíhá přes \textbf{TCP}.
  \item Všechny zprávy jsou \textbf{textové řádky} ukončené znakem \texttt{\textbackslash n}.
  \item Oddělovačem parametrů je \textbf{whitespace} (mezera / tabulátor).
  \item Tokeny vždy začínají prefixem \texttt{C45}.
  \item TCP je stream: zpráva může přijít po částech nebo „slepeně“; hranice zpráv určuje pouze \texttt{\textbackslash n}.
\end{itemize}

\subsection{Přenášené datové typy a validace}
\begin{itemize}[topsep=0.25em]
  \item \texttt{<name>} --- jméno hráče, neprázdné, bez whitespace, délka \(\le 63\) znaků (server má \texttt{MAX\_NAME\_LEN=64}).
  \item \texttt{<lobby>} --- číslo lobby (1..N); pro reconnect je povoleno \texttt{0} jako „neznámé“ (pokud není hráč ve hře, klient se vrátí do lobby listu).
  \item \texttt{<n>} --- počet lobby ve snapshotu (kladné celé číslo).
  \item \texttt{<pairs>} --- řetězec délky \(\texttt{2*n}\) tvořený číslicemi; dvojice \texttt{players} (0..2) a \texttt{status} (0/1).
  \item \texttt{<card>} --- karta jako dvouznakový kód \texttt{[A23456789TJQK][CDHS]} (např. \texttt{AS}, \texttt{7H}).
  \item \texttt{<sec>} --- sekundy (timeout tahu, timeout reconnectu).
  \item \texttt{<score>} --- celé číslo; \(-1\) znamená bust, jinak typicky 0..21.
\end{itemize}

\subsection{Omezení délky zpráv a důsledky pro návrh}
Server čte řádky do bufferu \texttt{READ\_BUF=256}; klient proto nesmí posílat extrémně dlouhé řádky (jinak hrozí \texttt{C45WRONG} / disconnect).

\subsection{Přehled zpráv}
\label{sec:protocol-messages}

Všechny zprávy jsou textové řádky zakončené \texttt{\textbackslash n}.

\subsubsection{Zprávy klient \(\to\) server}
\begin{itemize}[topsep=0.25em]
  \item \texttt{C45<name>} --- handshake (registrace jména).
  \item \texttt{C45REC <name> <lobby>} --- reconnect/resume session.
  \item \texttt{C45J <lobby>} --- join lobby.
  \item \texttt{C45B} --- „Back“ (návrat do lobby listu / refresh snapshotu; ve hře forfeit).
  \item \texttt{C45H} --- HIT (jen ve svém tahu).
  \item \texttt{C45S} --- STAND (jen ve svém tahu).
  \item \texttt{C45PO} --- odpověď na PING (\texttt{C45PI}).
\end{itemize}

\subsubsection{Zprávy server \(\to\) klient}
\begin{itemize}[topsep=0.25em]
  \item \texttt{C45OK} --- obecné potvrzení.
  \item \texttt{C45L <n> <pairs>} --- snapshot lobby listu (kompaktní 1 řádek).
  \item \texttt{C45REC\_OK} --- reconnect akceptován.
  \item \texttt{C45WRONG [detail]} --- chyba protokolu / nevalidní požadavek.
  \item \texttt{C45DOWN [reason]} --- server se vypíná; klient má ukončit spojení.
  \item \texttt{C45D <c1> <c2>} --- počáteční dvě karty.
  \item \texttt{C45C <card>} --- hráč si dobral kartu.
  \item \texttt{C45T <name> <sec>} --- kdo je na tahu a timeout v sekundách.
  \item \texttt{C45B <name> <value>} --- bust (hráč překročil 21).
  \item \texttt{C45TO} --- timeout lokálního hráče (auto-stand).
  \item \texttt{C45OD <name> <sec>} --- oponent se odpojil, server čeká \texttt{sec}.
  \item \texttt{C45OB <name>} --- oponent se vrátil, hra pokračuje.
  \item \texttt{C45R <p1> <s1> <p2> <s2> <winner>} --- výsledek hry (\texttt{winner} může být \texttt{PUSH}).
\end{itemize}

\subsubsection{Keep-alive}
\begin{itemize}[topsep=0.25em]
  \item \texttt{C45PI} --- PING (obousměrně).
  \item \texttt{C45PO} --- PONG (obousměrně).
\end{itemize}

\subsection{Poznámky}
Token \texttt{C45B} má dva významy: klient \(\to\) server \texttt{C45B} = „Back“, server \(\to\) klient \texttt{C45B <name> <value>} = bust (rozliší se směrem a parametry).

\subsection{Sekvenční scénáře}
\begin{itemize}[topsep=0.25em]
  \item \textbf{Connect + handshake:} TCP connect, klient pošle \texttt{C45<name>}, server odpoví \texttt{C45OK} a pošle \texttt{C45L}.
  \item \textbf{Lobby:} klient pošle \texttt{C45J <lobby>}; po úspěchu přijde \texttt{C45OK}. Klient může poslat \texttt{C45B} pro refresh \texttt{C45L}.
  \item \textbf{Game:} server pošle \texttt{C45D}, pak opakovaně \texttt{C45T}. Klient v tahu posílá \texttt{C45H} nebo \texttt{C45S}. Server posílá \texttt{C45C}, případně \texttt{C45B <name> <value>} a nakonec \texttt{C45R}.
  \item \textbf{Disconnect + reconnect:} při výpadku server pošle druhému hráči \texttt{C45OD}. Odpojený hráč se vrací přes \texttt{C45REC <name> <lobby>} a server odpoví \texttt{C45REC\_OK} a obnoví hru (snapshot ruky).
  \item \textbf{Back po hře:} po \texttt{C45R} klient odešle \texttt{C45B} a server pošle nový \texttt{C45L}.
\end{itemize}

\subsection{Relační diagram komunikace}
Ukazuje hlavní zprávy mezi klientem a serverem a typické stavy obou stran.
\begin{lstlisting}
Legenda:
  C->S  klient posila na server
  S->C  server posila na klienta
  [C:...] a [S:...] jsou zjednodusene stavy

1) CONNECT + HANDSHAKE + LOBBY LIST
  [C:DISCONNECTED] TCP connect
  [S:LISTEN] accept -> [S:HANDSHAKE]
  C->S: C45<name>
  S->C: C45OK
  S->C: C45L <n> <pairs>
  [C:LOBBY_CHOICE]    [S:LOBBY_SELECT]

2) JOIN LOBBY
  C->S: C45J <lobby>
  S->C: C45OK
  [C:WAIT_START]      [S:WAIT_FOR_GAME]

3) GAME (zjednodusene)
  S->C: C45D <c1> <c2>
  opakovane:
    S->C: C45T <name> <sec>
    pokud jsem na tahu:
      C->S: C45H | C45S
      S->C: C45C <card>        (po HIT)
      S->C: C45TO              (kdyz vyprsi cas)
      S->C: C45B <name> <val>  (bust)
  konec hry:
    S->C: C45R <p1> <s1> <p2> <s2> <winner>

4) DISCONNECT + RECONNECT (behem hry)
  pokud klient vypadne:
    S->oponent: C45OD <name> <sec>
    [C:RECONNECTING] nove TCP spojeni
    C->S: C45REC <name> <lobby>
    S->C: C45REC_OK
    S->C: snapshot ruky (C45D + 0..N * C45C) nebo C45L ...

5) KEEP-ALIVE (v libovolnem stavu)
  C<->S: C45PI / C45PO
\end{lstlisting}

\section{Chybové stavy a jejich hlášení}
\label{sec:errors}

\subsection{Protokolové chyby (\texttt{C45WRONG...})}
Server posílá \texttt{C45WRONG} typicky při:
\begin{itemize}[topsep=0.25em]
  \item Neplatném formátu zprávy (neznámý token, chybějící parametry).
  \item Neplatném lobby čísle nebo akci v nesprávné fázi.
  \item Neplatném reconnect requestu.
\end{itemize}

Chování po chybě závisí na fázi:
\begin{itemize}[topsep=0.25em]
  \item \textbf{Handshake / reconnect:} server ukončí spojení (nepřijatelný klient).
  \item \textbf{Lobby select:} server může poslat \texttt{C45WRONG} a zůstat připojený (uživatel může zadat jiné lobby).
  \item \textbf{Po skončení hry:} neznámé zprávy jsou považovány za chybu a spojení je ukončeno.
\end{itemize}

\subsection{Kolize jmen (\texttt{C45WRONG NAME\_TAKEN})}
Jméno hráče je unikátní napříč aktivními spojeními. Pokud hráč zvolí jméno, které je již rezervované, server odpoví:
\begin{lstlisting}[language=C45proto]
C45WRONG NAME_TAKEN
\end{lstlisting}
Klient tuto situaci zobrazí uživateli a ukončí session (nový connect je nutný s jiným jménem).

\subsection{Server shutdown (\texttt{C45DOWN})}
Pokud se server vypíná (např. SIGINT nebo ztráta bind IP), pošle všem připojeným klientům:
\begin{lstlisting}[language=C45proto]
C45DOWN [reason]
\end{lstlisting}
Klient má ukončit spojení a přejít na úvodní obrazovku.

\section{Implementace serveru (programátorská dokumentace)}
\label{sec:server-impl}

\subsection{Moduly a odpovědnosti}
\begin{itemize}[topsep=0.25em]
  \item \texttt{src/main.c} --- načte \texttt{config.txt}, zpracuje CLI volby, inicializuje lobbyny a spustí server.
  \item \texttt{src/server.c} --- accept-loop, jedno vlákno na klienta, handshake, lobby select, koordinace s herním vláknem, registry aktivních jmen.
  \item \texttt{src/game.c} --- logika hry, správa lobby poolu, herní vlákno pro každé lobby, reconnect během hry.
  \item \texttt{src/protocol.c} --- robustní \texttt{read\_line}/\texttt{write\_all} a serializace snapshotu lobby (\texttt{C45L}).
\end{itemize}

\subsection{Klíčové funkce a jejich role}
\begin{itemize}[topsep=0.25em]
  \item \texttt{run\_server(ip,port)} --- listen socket, \texttt{poll()} na accept, kontrola dostupnosti bind IP, vytvoření per-client vláken; při konci pošle \texttt{C45DOWN}.
  \item \texttt{client\_thread(...)} --- stavový automat klienta (handshake/reconnect, lobby, čekání, návrat po hře); toleruje \texttt{C45PI/C45PO}.
  \item \texttt{send\_lobbies\_snapshot(fd)} --- vytvoří a odešle snapshot \texttt{C45L}.
  \item \texttt{start\_game\_if\_ready(i)} --- spustí herní vlákno lobby po naplnění 2 hráči.
  \item \texttt{lobby\_game\_thread(...)} --- běh hry (tahy, timeouty, reconnect, výsledky).
  \item \texttt{read\_line()/write\_all()} --- bezpečné řádkové I/O nad TCP.
\end{itemize}

\subsection{Síť a odolnost}
Server používá \texttt{poll()} (bez aktivního čekání) a keep-alive \texttt{C45PI/C45PO}. Při shutdownu (nebo ztrátě bind IP) informuje klienty přes \texttt{C45DOWN} a ukončí spojení.

\subsection{Paralelizace}
Server používá \textbf{vlákna} (\texttt{pthread}):
\begin{itemize}[topsep=0.25em]
  \item \textbf{Accept-loop} běží v hlavním vlákně a pro každého klienta spouští \texttt{client\_thread()}.
  \item \textbf{Per-client vlákno} obsluhuje stavový automat jednoho hráče (handshake, lobby, čekání, návrat).
  \item \textbf{Herní vlákno lobby} je spuštěno, když lobby dosáhne kapacity 2 hráčů (\texttt{start\_game\_if\_ready()}).
\end{itemize}

Synchronizace probíhá pomocí mutexů:
\begin{itemize}[topsep=0.25em]
  \item \texttt{Lobby.mtx} --- chrání stav lobby (hráči, hand, fd, is\_running).
  \item \texttt{g\_names\_mtx} --- chrání globální registry aktivních jmen a jejich tokenů.
  \item \texttt{g\_clients\_mtx} --- seznam připojených socketů (pro rychlé odpojení všech klientů).
\end{itemize}

\subsection{Obsluha reconnectu na serveru}
\begin{itemize}[topsep=0.25em]
  \item Klient se vrací přes \texttt{C45REC <name> <lobby>} na novém TCP spojení.
  \item Server odpoví \texttt{C45REC\_OK} a pokračuje buď obnovou hry (snapshot ruky), nebo návratem do lobby listu (\texttt{C45L}).
\end{itemize}

\subsection{Konfigurace serveru}
Server čte \texttt{server/config.txt} (key-value, whitespace oddělené). Příklad:
\begin{lstlisting}
LOBBY_COUNT 3
IP 0.0.0.0
PORT 10000
\end{lstlisting}

\section{Implementace klienta}
\label{sec:client-impl}

\subsection{Rozvrstvení a třídy}
\begin{itemize}[topsep=0.25em]
  \item \texttt{MainApp}:
    \begin{itemize}[topsep=0.25em]
      \item JavaFX entry point, vytváří scény (connect, name, lobby, waiting, game).
      \item Vytváří instanci \texttt{NetClient} a registruje \texttt{ProtocolListener}.
      \item Přepíná UI podle příchozích zpráv.
    \end{itemize}
  \item \texttt{NetClient}:
    \begin{itemize}[topsep=0.25em]
      \item Udržuje socket, \texttt{BufferedReader} a \texttt{OutputStream}.
      \item Spouští daemon reader thread, který provádí blokující \texttt{readLine()}.
      \item Implementuje parser protokolu a callbacky do \texttt{ProtocolListener}.
      \item Implementuje keep-alive a automatický reconnect.
    \end{itemize}
  \item \texttt{ProtocolListener}:
    \begin{itemize}[topsep=0.25em]
      \item Rozhraní callbacků pro UI (OK, snapshot lobby, herní eventy, chyby, reconnect progress).
    \end{itemize}
  \item \texttt{GameView}:
    \begin{itemize}[topsep=0.25em]
      \item Herna scéna: log, skóre, tlačítka HIT/STAND, tlačítko „Back to Lobby“.
      \item Zobrazuje reconnect status label přímo v herním okně.
      \item Při obnově hry (snapshot ruky) nezdvojuje log karet.
    \end{itemize}
\end{itemize}

\subsection{UI}
Klient má obrazovky pro připojení (IP/port), zadání jména, výběr lobby a hru. Ruční \texttt{Reconnect} je dostupný až po předchozím úspěšném připojení (klient zná poslední \texttt{name} a \texttt{lobby}).

\subsection{Paralelizace klienta}
Klient používá \textbf{dvě hlavní vlákna}:
\begin{itemize}[topsep=0.25em]
  \item JavaFX Application Thread --- rendering UI a zpracování akcí uživatele.
  \item Reader thread (\texttt{net-reader}) --- čte TCP stream, parsuje protokol a vyvolává callbacky.
\end{itemize}
Veškeré změny UI jsou prováděny pomocí \texttt{Platform.runLater(...)}.

\subsection{Keep-alive a reconnect }
\label{sec:client-reconnect}
Klient pravidelně ověřuje, že server odpovídá (PING/PONG). Při výpadku se pokusí automaticky obnovit spojení; při úspěchu pošle \texttt{C45REC <name> <lobby>} a pokračuje ve hře nebo se vrátí do lobby listu. Pokud auto-reconnect selže, uživatel použije ruční \texttt{Reconnect}.

\section{Požadavky na překlad, spuštění a běh aplikace}
\label{sec:build-run}

\subsection{Server (C)}
\textbf{Požadavky:}
\begin{itemize}[topsep=0.25em]
  \item Linux/Unix-like prostředí s POSIX API
  \item \texttt{gcc} nebo kompatibilní C11 překladač
  \item \texttt{make}
\end{itemize}

\textbf{Postup překladu:}
\begin{lstlisting}[language=bash]
cd server
make -j
\end{lstlisting}

\textbf{Spuštění:}
\begin{itemize}[topsep=0.25em]
  \item výchozí: server čte \texttt{config.txt} v aktuální složce,
  \item volitelně lze zadat IP a port přes CLI: \texttt{-i <IP> -p <PORT>}.
\end{itemize}
\begin{lstlisting}[language=bash]
./blackjack_server -i 0.0.0.0 -p 10000
\end{lstlisting}

\subsubsection{Konfigurace serveru (\texttt{config.txt})}
Konfigurační soubor je textový, každá položka je ve formátu \texttt{KEY VALUE} (oddělené whitespace). Podporované klíče:
\begin{itemize}[topsep=0.25em]
  \item \texttt{LOBBY\_COUNT} --- počet lobby (1--99),
  \item \texttt{IP} --- bind adresa (např. \texttt{0.0.0.0} nebo \texttt{127.0.0.1}),
  \item \texttt{PORT} --- port (1..65535).
\end{itemize}

\subsubsection{Logování}
Server vypisuje stavové informace na standardní výstup (\texttt{stdout}): připojení klientů, protokolové chyby, start/stop hry a informace o odpojení/reconnectu.

\subsection{Klient (JavaFX)}
\textbf{Požadavky:}
\begin{itemize}[topsep=0.25em]
  \item Java JDK 17
  \item Maven
\end{itemize}

\textbf{Postup překladu/spuštění:}
\begin{lstlisting}[language=bash]
cd clientBlackJack/client
mvn clean javafx:run
\end{lstlisting}

\subsubsection{Poznámky k běhu JavaFX}
Spuštění přes \texttt{mvn javafx:run} automaticky nastaví potřebné JavaFX moduly a knihovny. V případě ručního spouštění mimo Maven je potřeba správně nastavit \texttt{--module-path} a \texttt{--add-modules} pro JavaFX (nebo vytvořit bundlovanou distribuci).

\textbf{Běh aplikace:}
\begin{itemize}[topsep=0.25em]
  \item Uživatel zadá IP a port serveru.
  \item Zadá jméno (bez mezer).
  \item Zobrazí se lobby list, uživatel zvolí lobby číslo.
  \item Po startu hry se zobrazí herní okno.
\end{itemize}

\subsubsection{Ruční reconnect (uživatelský postup)}
Pokud automatický reconnect selže a aplikace zobrazí hlášku „Please reconnect manually“, uživatel může:
\begin{itemize}[topsep=0.25em]
  \item přejít na úvodní obrazovku (Connect),
  \item zadat IP/port,
  \item použít tlačítko \textbf{Reconnect} (je dostupné pouze pokud klient zná poslední \texttt{name} a \texttt{lobby}).
\end{itemize}
Tlačítko odešle \texttt{C45REC <name> <lobby>} a klient pokračuje v čekání na obnovení hry nebo na lobby snapshot.

\section{Ukázky komunikace}
\subsection{Handshake a lobby snapshot}
\begin{lstlisting}[language=C45proto]
# Client -> Server
C45Alice

# Server -> Client
C45OK
C45L 3 000000
\end{lstlisting}

\subsection{Join lobby a start hry}
\begin{lstlisting}[language=C45proto]
# Client -> Server
C45J 1

# Server -> Client
C45OK

# pozdeji, kdyz jsou v lobby 2 hraci:
C45D AS 5S
C45T Alice 60
\end{lstlisting}

\subsection{Reconnect během hry}
\begin{lstlisting}[language=C45proto]
# Client -> Server (nove TCP spojeni)
C45REC Alice 1

# Server -> Client
C45REC_OK

# snapshot ruky (minimalne prvni 2 karty)
C45D AS 5S
C45C 3H

# pokracovani hry
C45T Bob 60
\end{lstlisting}

\subsection{Poznámky}
\begin{itemize}[topsep=0.25em]
  \item Typické chyby serveru: \texttt{C45WRONG} (nevalidní požadavek), \texttt{C45WRONG NAME\_TAKEN} (kolize jmen), \texttt{C45DOWN} (server se vypíná).
  \item Důležité timeouty: tah hráče je standardně 60 s (\texttt{TURN\_TIMEOUT\_SEC}), server čeká na reconnect ve hře 60 s (\texttt{RECONNECT\_TIMEOUT\_SEC}).
\end{itemize}

\section{Závěr a zhodnocení dosažených výsledků}

Byla úspěšně implementována síťová hra Blackjack ve variantě 1 vs. 1, která je rozdělena na serverovou část v jazyce C a klientskou část v Javě (JavaFX). Cílem bylo vytvořit funkční klient/server řešení s vlastním protokolem a současně zajistit použitelnost i při běžných síťových problémech (zpoždění, fragmentace, výpadek spojení).

\subsection{Zhodnocení protokolu}
Zvolený protokol je záměrně jednoduchý: textové řádky přes TCP, tokeny s prefixem \texttt{C45} a parametry oddělené whitespace. Tento přístup má výhody v laditelnosti (lze testovat přes \texttt{nc}) a v možnosti snadno vytvořit alternativního klienta/server.
Součástí protokolu jsou:
\begin{itemize}[topsep=0.25em]
  \item handshake a lobby snapshot (\texttt{C45OK}, \texttt{C45L}),
  \item průběh hry (deal/turn/card/bust/timeout/result),
  \item keep-alive (\texttt{C45PI/C45PO}) pro rychlejší detekci „tichých“ výpadků,
  \item reconnect (\texttt{C45REC}, \texttt{C45REC\_OK}) pro navázání na běžící session.
\end{itemize}

\subsection{Zhodnocení implementace}
Server používá POSIX API (sockety, \texttt{poll()}, \texttt{pthread}) a běží jako vícevláknová aplikace: jedno vlákno na klienta a herní vlákno pro lobby, jakmile jsou připojeni dva hráči. Klient používá JavaFX pro UI a samostatné „reader“ vlákno pro blokující čtení ze socketu; změny UI jsou prováděny přes \texttt{Platform.runLater}.

\subsection{Omezení a možnosti zlepšení}
V aktuální verzi jsou nejdůležitější omezení a možné další kroky:
\begin{itemize}[topsep=0.25em]
  \item \textbf{Škálování lobby listu:} lobby snapshot \texttt{C45L} je jednorázová zpráva; pro tisíce lobby je vhodnější stránkování (více zpráv) nebo jiný formát.
  \item \textbf{Bezpečnost:} protokol je bez autentizace a šifrování (TLS); je vhodný pro LAN/testování, ne pro veřejné nasazení.
  \item \textbf{Herní rozšíření:} přidání dealera, více hráčů, botů nebo statistik by vyžadovalo rozšíření protokolu i UI.
  \item \textbf{Testování:} přidání automatizovaných integračních testů by zlepšilo jistotu při úpravách (např. testy reconnectu a chybových stavů).
\end{itemize}

\end{document}
